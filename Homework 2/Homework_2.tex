\documentclass{article}
\usepackage[margin=0.9in]{geometry}
\usepackage{fancyhdr}
\usepackage{amsmath}
\usepackage{graphicx}
\pagestyle{fancy}
\lhead{Math 155 \\Homework \#2}
\rhead{Joshua Lai \\804-449-134}

\begin{document}

\begin{enumerate}
\item[1)] Give a single intensity transformation function $T$ for spreading the intensities of an
image so the lowest intensity is 0 and the highest is $L-1$.

Let the image be represented as a function $f_0$ and let the maximum and minimum be represented by the
notation $f_{max}$ and $f_{min}$, respectively.  The new minimum and maximum of the transformed image,
denoted as $f_t$ will be $0$ and $L-1$, respectively.  Thus, we can see that the transformation function is:

\begin{equation}
f_t(x,y)=\frac{L-1}{f_{max}-f_{min}}[f_0-f_{min}]
\end{equation}

\item[2)] (Histogram equalization in continuous variables) An image has the gray-level PDF

\begin{gather*}
p_r(r)	=\begin{cases}
\frac{6r+2}{3(L-1)^2 + 2(L-1)}\\0, \text{ otherwise}
\end{cases}
\end{gather*}

if $0 \leq r \leq L - 1$ with $L - 1 > 0$.
\begin{enumerate}
\item[(a)] Verify some of the properties that a PDF has to satisfy: $p_r(r) \geq 0$ for all $r \in (-\infty, \infty)$ and $\int_{-\infty}^{\infty} p_r(r)dr=1$.
\item[~]
By defintion of probability density function (PDF), $p_r(r)$ is such that 

\begin{equation}
\int_{a}^{b} p_r(r)dr = P(X \in [a, b])
\end{equation}

for any interval $[a,b]$. Since it impossible for probabilites to be negative, we can write $P(X \in [a, b]) \geq 0$, and thus, we know that

\begin{equation}
\int_{a}^{b} p_r(r)dr \geq 0
\end{equation}

\begin{equation}
\Rightarrow p_r(r) \geq 0
\end{equation}

for any interval $[a,b]$. But, the above integral can only be non-negative for all intervals $[a,b]$ only if the integrand function itself is non-negative, that is, $p_r(r) \geq 0$
for all $x$.  Thus, the first property is true. \newline
Next, by the definition of probability, the sum of all of the probability of the events must be 1. Thus, we can write

\begin{equation}
1=P(X \in (-\infty, \infty))=\int_{-\infty}^{\infty} p_r(r)dr=1.
\end{equation}

\item[(b)] Find the tranformation function $s = T(r)$ obtained through “histogram equalization”
in continuous variables.

\begin{equation}
s=T(r)=(L-1) \int_{0}^{r} \frac{6r+2}{3(L-1)^2 + 2(L-1)}dr
\end{equation}

\begin{equation} 
\Rightarrow \frac{2}{3(L-1)+2} \int_{0}^{r} (3r+1) dr
\end{equation}

\begin{equation}
\Rightarrow \frac{2}{3(L-1)+2} \times (\frac{3r^2}{2} + r)
\end{equation}

\begin{equation}
\Rightarrow s = T(r) = \frac{3r^2 + 2r}{3(L-1)+2}
\end{equation}

\item[(c)] Verify that $p_s(s)$ is a uniform "flat" distribution for $s \in [0, 1]$ (recall the formula $p_s(s) = p_r(r) \left| \frac{dr}{ds} \right|$ ).

\begin{equation}
p_s(s) = p_r(r) \left| \frac{dr}{ds} \right| = \frac{6r+2}{3(L-1)^2 + 2(L-1)} \left| \left[ \frac{d}{dr} \frac{3r^2 + 2r}{3(L-1)+2} \right]^{-1} \right|
\end{equation}

\begin{equation}
\Rightarrow \frac{6r+2}{3(L-1)^2 + 2(L-1)} \left| \left[ \frac{6r+2}{3(L-1)+2} \right]^{-1} \right|
\end{equation}

\begin{equation}
\Rightarrow \frac{1}{L-1}
\end{equation}

Since $r$ is nonnegative and we assume that $L > 1$, we can see that the result is a uniform PDF.
\end{enumerate}

\item[3)] (Histogram matching in continuous variables) An image has the gray-level PDF
$p_r(r) = -2r + 2$, with $0 \leq r \leq 1$. It is desired to transform the gray levels of this image
so that they will have the specified $p_z(z) = 2z$, $0 \leq z \leq 1$. Assume continuous quantities
and find the transformation (in terms of $r$ and $z$) that will accomplish this (here $L-1 = 1$).

First, we find the the histogram equalization transformation:
\begin{equation} 
s=T(r)=(L-1) \int_{0}^{r} (-2r+2)dr = (L-1)(-r^2 + 2r) .
\end{equation}

Now, we look for the image with a specified histogram:

\begin{equation}
G(z) = (L-1) \int_{0}^{z} 2z dz = (L-1)z^2 .
\end{equation}

Now, we require $G(z) = s$, but 

\begin{equation}
G(z) = (L-1) z^2,
\end{equation}

so, now we can write

\begin{equation}
s=(L-1) z^2.
\end{equation}

Next, we will solve for $z$:

\begin{equation}
z=\sqrt{\frac{s}{(L-1)}}
\end{equation}

Now, we can generate the $z$'s directly from the intensities, $r$, of the input image:

\begin{equation}
z= \sqrt{(-r^2 + 2r)}
\end{equation}

\item[4)] A linear spatial filter of size $(2a+1) \times (2b+1)$ defined by the transformation $H$, $g=H[f]$, is given by

\begin{equation}
g(x,y)=\sum_{s=-a}^{a} \sum_{t=-b}^{b} w(s,t)f(x+s, y+t),
\end{equation}

where $f(x, y)$ is a given input image, $a, b$ are positive integers, and $w(s, t)$ are weights for $-a \leq s \leq a$, $-b \leq t \leq b$.

\begin{enumerate}
\item[(a)] Give the definition of a linear transformation $H : V \rightarrow V$, where $V$ is a vector space.

Looking back at some linear algebra materal (Math 115A), we know that a transformation $H$ is linear if for all $m,n \in V$ and $c \in F$,
we have that:

\begin{enumerate}
\item[1.] $H(m + n) = H(m) + H(n)$ \quad (vector-addition)
\item[2.] $H(cm)=cH(m)$ \quad\quad\kern 0.45in (scalar-multiplication)
\end{enumerate}

\item[(b)]  Show that $H$ defined above is indeed a linear transformation (assume images defined
on the entire plane, or ignore border effects).

To prove that $H$ is indeed a linear transformation, we need to prove that the two properties above hold throughout the transformation.

\begin{equation}
H(cm + n) = \sum_{s=-a}^{a} \sum_{t=-b}^{b} w(s,t)(cm+n)(x+s, y+t)
\end{equation}

\begin{equation}
\Rightarrow \sum_{s=-a}^{a} \sum_{t=-b}^{b} \left[ w(s,t)(cm)(x+s,y+t) + w(s,t)(n)(x+s,y+t) \right]
\end{equation}

\begin{equation}
\Rightarrow \sum_{s=-a}^{a} \sum_{t=-b}^{b} w(s,t)(cm)(x+s,y+t) + \sum_{s=-a}^{a} \sum_{t=-b}^{b} w(s,t)(n)(x+s,y+t)
\end{equation}

\begin{equation}
\Rightarrow c\sum_{s=-a}^{a} \sum_{t=-b}^{b} w(s,t)m(x+s,y+t) + \sum_{s=-a}^{a} \sum_{t=-b}^{b} w(s,t)n(x+s,y+t)
\end{equation}

\begin{equation}
\Rightarrow cH(m) + H(n)
\end{equation}

\end{enumerate}
\end{enumerate}
\end{document}