\documentclass{article}
\usepackage[margin=1.0in]{geometry}
\usepackage{fancyhdr}
\usepackage[]{mcode}
\usepackage{amsmath, amssymb}
\pagestyle{fancy}
\lhead{Math 155 \\Homework \#9}
\rhead{Joshua Lai \\804-449-134}

\begin{document}

\begin{enumerate}
\item[1)] Consider the motion blur in the frequency domain given by 
$$H(u,v)=\int_{0}^Te^{-2\pi i[ux_0(t)+vy_0(t)]}dt \hspace{0.1in}.$$ 
For uniform motion given by $x_0(t)=\frac{at}{T}$ and $y_0(t)=\frac{bt}{T}$ (where $T$ is exposure time), show that the degradation function becomes 
$$H(u,v)=\frac{T}{\pi(ua+vb)}\sin{[\pi(ua+vb)]}e^{-\pi i(ua+vb)} \hspace{0.1in}.$$

\textbf{Answer: }We start with our given equation and substitute in $x_0(t)$ and $y_0(t)$ to get:

\begin{equation}
H(u,v) = \int_{0}^{T} e^{\frac{-i2\pi t}{T}  (ua + vb) } dt \hspace{0.1in} .
\end{equation}

\begin{equation}
\Rightarrow - \frac{T}{i2\pi (ua + vb)} \left[ e^{\frac{-i2\pi t}{T}  (ua + vb)} \right] \Big|_0^T
\end{equation}

\begin{equation}
\Rightarrow - \frac{T}{i2\pi (ua + vb)} \left[ e^{-i2\pi (ua + vb)} - 1 \right]
\end{equation}

\begin{equation}
\Rightarrow \frac{T}{i2\pi (ua + vb)} \left[ 1 - e^{-i2\pi (ua + vb)} \right]
\end{equation}

\begin{equation}
\Rightarrow \frac{T}{i2\pi (ua + vb)} \left[ e^{-i\pi(ua + vb)}e^{i\pi(ua + vb)} - e^{-i2\pi (ua+vb)} \right]
\end{equation}

\begin{equation}
\Rightarrow \frac{T}{i2\pi (ua + vb)} \left[ e^{-i\pi (ua + vb)} \right] \left[ e^{i\pi(ua + vb)} - e^{-i\pi(ua + vb)} \right]
\end{equation}

\begin{equation}
\Rightarrow \frac{T}{i2\pi (ua + vb)} \left[ e^{-i\pi (ua + vb)} \right] (i2 \sin(\pi (ua + bv)))
\end{equation}

\begin{equation}
\Rightarrow \frac{T}{\pi(ua+vb)}\sin{[\pi(ua+vb)]}e^{-\pi i(ua+vb)} \hspace{0.1in} .
\end{equation}

\item[2)] Recall the 1D Laplacian of Gaussian ($L \circ G$) operator 
$$\nabla^2 h(r)= \Big[\frac{r^2-\sigma^2}{\sigma^4}\Big]e^{-\frac{r^2}{2\sigma^2}} \hspace{0.1in}.$$ 

\begin{enumerate}
\item[a)] Show that the integration of the $L \circ G$ operator $\nabla^2 h$ is zero. Use the following identities: 
$$\frac{1}{\sqrt{2\pi\sigma^2}}\int_{-\infty}^{+\infty}e^{-r^2/(2\sigma^2)}dr=1, \ \sigma^2=\frac{1}{\sqrt{2\pi\sigma^2}}\int_{-\infty}^{+\infty}r^2e^{-r^2/(2\sigma^2)}dr \hspace{0.1in} .$$

\textbf{Answer: }We need to take the integration of the given formula, thus we get the following:

\begin{equation}
\int_{-\infty}^{\infty} \nabla^2 h(r) = \int_{-\infty}^{\infty} \Big[\frac{r^2-\sigma^2}{\sigma^4}\Big]e^{-\frac{r^2}{2\sigma^2}} dr
\end{equation}

\begin{equation}
\Rightarrow \frac{1}{\sigma ^ 4} \int_{-\infty}^{\infty} r^2 e^{-\frac{r^2}{2\sigma^2}} dr - \int_{-\infty}^{\infty} \frac{1}{\sigma ^ 2} e^{-\frac{r^2}{2\sigma^2}} dr
\end{equation}

Since we have the given identities, we know that 

\begin{equation}
\frac{1}{\sqrt{2\pi\sigma^2}}\int_{-\infty}^{+\infty}e^{-r^2/(2\sigma^2)}dr = 1 \hspace{0.1in} \Rightarrow \hspace{0.1in} \int_{-\infty}^{\infty} e^{-\frac{r^2}{2 \sigma^2}} dr = \sqrt{2 \pi} \sigma
\end{equation}

and

\begin{equation}
\ \sigma^2=\frac{1}{\sqrt{2\pi\sigma^2}}\int_{-\infty}^{+\infty}r^2e^{-r^2/(2\sigma^2)} dr \hspace{0.1in} \Rightarrow \hspace{0.1in} \sqrt{2 \pi} \sigma ^ 3 = \int_{-\infty}^{\infty} r^2 e^{-\frac{r^2}{2 \sigma ^ 2}}
\end{equation}

and therefore, we can write the following:

\begin{equation}
\Rightarrow \left( \frac{\sqrt{2 \pi} \sigma ^ 3}{\sigma ^ 4} \right) \left( \sqrt{2 \pi} \sigma \right) + \left( \frac{\sqrt{2 \pi} \sigma ^ 3}{\sigma ^ 4} \right) \left( \sqrt{2 \pi} \sigma \right) - \frac{2(\sqrt{2 \pi} \sigma) ^ 2}{\sigma ^ 2}
\end{equation}

\begin{equation}
\Rightarrow 2\pi + 2 \pi - 4 \pi
\end{equation}

\begin{equation}
\Rightarrow 0
\end{equation}

Thus, the integration of the $L \circ G$ operator $\nabla^2 h$ is zero.

\item[b)] Prove that the integration of any function convolved with this operator is also zero. (Hint: Use the frequency domain.)

\textbf{Answer: }Let the Fourier transform of $\nabla^2 h$ be $\nabla^2 \hat h$. Let the image and its Fourier transform be denoted as $f(x,y)$ and $F(u,v)$ respectively. The expression $f(x,y)$ convolved with this operator
is 

$$h(x,y) = f(x,y) \star \nabla^2 h \hspace{0.1in} .$$

Convolution in the time domain leads to multiplication in the frequency domain:

$$H(u,v) = F(u,v) \nabla^2 h \hspace{0.1in}$$

The integration of $\nabla^2 h$ is zero. Also, the integration in the time domain is equivalent to the value of its transform at the origin of the frequency domain.  Similarly, the integration of $h(x,y)$ is equivalent to $H(0,0)$.
Thus, we can write:

\begin{equation}
\overline{h}(x,y) = H(0,0)
\end{equation}

\begin{equation}
\Rightarrow F(0,0)\nabla^2 h(0,0)
\end{equation}

\begin{equation}
\Rightarrow 0
\end{equation}

Thus, the integration of any image convolved with this operator is zero.

\end{enumerate}

\newpage
\textbf{Optional Problems:}

\item[3)] \textbf{Parametric Wiener Filter}
\begin{enumerate}
\item[a)] Implement a motion blurring filter as in problem (1).

\begin{lstlisting}
%Import Image
original_image = imread('Fig5.26a.jpg');
original_image_double = im2double(original_image);

[M,N] = size(original_image_double);
[V,U] = meshgrid(1:N, 1:M);

U = U - floor(M / 2);
V = V - floor(N / 2);

%Define Parameters
a = 0.1;
b = 0.1;
T = 1;

exponential = pi*((U * a) + (V * b) + eps);

degradation_function = T ./ exponential .* sin( exponential ) ...
    .* exp(-1j * exponential);
degradation_function = ifftshift(degradation_function);

frequency_shift = fft2(original_image_double);
applied_filter = degradation_function .* frequency_shift;
new_image = real(ifft2(applied_filter));
new_image = im2uint8(mat2gray(new_image));
\end{lstlisting}

\item[b)] Download `Fig. 5.26(a)' and blur the image in the $+45 ^{\circ}$ direction using $T=1$, as in `Fig. 5.26(b)' ($a = b = 0.1$).
\newline\textbf{Please see attached pages for images.}

\item[c)] Add a small amount of Gaussian noise of $0$ mean to the blurred image.

\begin{lstlisting}
B = imnoise(new_image, 'gaussian', 0);
\end{lstlisting}

\item[d)] Restore the image using the parametric Wiener filter given by:
$$\hat F(u,v)=\Big[\frac{1}{H(u,v)}\frac{|H(u,v)|^2}{|H(u,v)|^2+K}\Big]G(u,v),$$
where $K$ is a specified constant, chosen to obtain best visual results.

\textbf{Please see attached pages for images.} \newline
Note: Some of the code was taken from StackOverflow.
 
\begin{lstlisting}
PSF = fspecial('motion', 21, 11);
blurred = imfilter(original_image, PSF, 'conv', 'circular');
noise_mean = 0;
noise_var = 0.0001;
blurred_noisy = imnoise(blurred, 'gaussian', noise_mean, noise_var);
signal_var = var(original_image_double(:));
wiener_filter = deconvwnr(blurred_noisy, PSF, noise_var / signal_var);
imshow(wiener_filter); title('Wiener Filter');
\end{lstlisting}

\end{enumerate}

\newpage
\item[4a)] Give the main steps of edge finding using the zero-crossings of the Laplacian of
Gaussian.

\textbf{Answer: }We first the import the image as $f(x,y)$ and we define $h_G$ as our Gaussian (blur) function. We then convolve the image with the blur function and take the Laplacian of the image:
$$\nabla ^2(h_G \star f) = (\nabla ^ 2 h_G) \star f \hspace{0.1in} .$$

We know that $ (\nabla ^ 2 h_G)$ is the Laplcian of Gaussian operator. Afterwards, we take the values after the convolution such as if the value at a certain pixel is greater than 0, then it will be assigned black and otherwise, it will be assigned white.
Through that, we can see the boundaries of the image by using a derivative mask such as the gradient, and thus, we have the zero crossing method.

\item[4b)] Implement and apply this method to the angiogram image from `Fig. 10.15(a)' (as in`Fig. 10.22' for the house image, but you will apply your method to the angiogram image).\newline
\textbf{Please see attached pages for images.}

\begin{lstlisting}
original_image = imread('Fig10.15a.jpg');
new_image = edge(original_image, 'zerocross');
imshow(new_image); title('Zero-Crossing of Image');
\end{lstlisting}

\end{enumerate}

\end{document}