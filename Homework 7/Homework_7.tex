\documentclass{article}
\usepackage[margin=1.0in]{geometry}
\usepackage{fancyhdr}
\usepackage[]{mcode}
\usepackage{amsmath, amssymb}
\usepackage{graphicx}
\pagestyle{fancy}
\lhead{Math 155 \\Homework \#7}
\rhead{Joshua Lai \\804-449-134}

\begin{document}

\begin{enumerate}

\item[1)] Consider the two-dimensional case.
\begin{enumerate}
\item[a)] Write the Inverse Fourier Transform formula in 2D (express $f(x,y)$ function of $F(u,v)$ in the continuous case). 

\textbf{Answer: }The Inverse Fourier Transform formula in two dimensions is:
$$f(x,y) = \int_{-\infty}^{\infty} \int_{-\infty}^{\infty} F(u,v) e^{i2 \pi (ux + vy)} du dv \hspace{0.1in} .$$

\item[b)] Assume $f$ is twice differentiable. Using (a), find the Fourier transform of the mixed partial derivative $\frac{\partial^2 f}{\partial x\partial y}$, function of $F$.
\end{enumerate}

We first will take the partial derivative $\frac{\partial f}{\partial x}$ of the function inverse Fourier Transform:

\begin{equation}
\frac{\partial f}{\partial x} = i2\pi u e^{i2 \pi (ux + vy)} \int_{-\infty}^{\infty} \int_{-\infty}^{\infty} F(u,v) du dv \hspace{0.1in} .
\end{equation}

We then take the parital derivative $\frac{\partial}{\partial y}$ of the equation $\frac{\partial f}{\partial x}$ and we get:

\begin{equation}
\frac{\partial ^2 f}{\partial x \partial y} = (i2\pi )^2 uv e^{i2 \pi (ux + vy)} \int_{-\infty}^{\infty} \int_{-\infty}^{\infty} F(u,v) du dv \hspace{0.1in} .
\end{equation}

\item[2)] We have seen in continuous variables that ${\cal F}(\delta)\equiv 1$, thus $\delta$ and $1$ form a Fourier pair; we also have that ${\cal F}(1)=\delta$. Using the second property 
and the translation property, show that the Fourier transform of $f(x)=\sin(2\pi u_0x)$, where $u_0$ is a real number, is 
$$F(u)=(i/2)\Big[\delta(u+u_0)-\delta(u-u_0)\Big] \hspace{0.1in}.$$
(Hint: You could express  $f$ function of exponentials.)

\textbf{Answer: }We first apply the Fourier transformation to the function given:
$${\cal F}[f(x)] = \int_{-\infty}^{\infty} \sin(2\pi u_0 x) e^{-i2\pi (ux)} dx \hspace{0.1in}.$$

We then convert the sine function into exponentials:
$$\sin\theta = \frac{e^{i\theta} - e^{-i\theta}}{2i}$$

and substitute it into our formula, which gives us the following:

\begin{equation}
\int_{-\infty}^{\infty} \frac{-i}{2} \left( e^{i2\pi (u_0 x)} - e^{-i2\pi (u_0 x)}\right) e^{-i2\pi (ux)} dx
\end{equation}

\begin{equation}
\Rightarrow \frac{-i}{2} \int_{-\infty}^{\infty} (1) e^{-i2\pi (ux)} e^{i2\pi (u_0 x)} dx - \left( \frac{-i}{2}\right) \int_{-\infty}^{\infty} (1) e^{-i2\pi (ux)} e^{-i2\pi (u_0 x)} dx
\end{equation}

Using the translation property of the Fourier Transform and the delta function, we can then write:
\begin{equation}
\Rightarrow \frac{-i}{2} {\cal F}(1) (u - u_0) - \left( \frac{-i}{2} \right) {\cal F} (1) (u + u_0)
\end{equation}

\begin{equation}
\Rightarrow \frac{i}{2} \left[ \delta(u + u_0) - \delta(u - u_0) \right]
\end{equation}

\item[3)] We have seen in continuous variables that ${\cal F}(\delta)\equiv 1$, thus $\delta$ and $1$ form a Fourier pair; we also have that 
${\cal F}(1)=\delta$. Using the second property and the translation property, show that the Fourier transform of the continuous function $f(x,y)=A\sin(2\pi u_0 x+2\pi v_0y)$ is 
$$F(u,v)=A\frac{i}{2}\Big[\delta(u+u_0,v+v_0)-\delta(u-u_0,v-v_0)\Big] \hspace{0.1in}.$$ 
(Hint: You could express $f$ as a function of exponentials)

\textbf{Answer: }We first apply the Fourier transformation to the function given:
$${\cal F}[f(x)] = \int_{-\infty}^{\infty} \int_{-\infty}^{\infty} A\sin(2\pi u_0 x + 2\pi v_0 y) e^{-i2\pi (ux + vy)} dx dy\hspace{0.1in}.$$

We then convert the sine function into its exponentials:

$$\sin\theta = \frac{e^{i\theta} - e^{-i\theta}}{2i}$$

which gives us:

\begin{equation}
\frac{-Ai}{2} \int_{-\infty}^{\infty} \int_{-\infty}^{\infty} \left[ e^{i(2\pi u_0 x + 2\pi v_0 y)} - e^{-i(2\pi u_0 x + 2\pi v_0 y)} \right] e^{-i2\pi (ux + vy)} dx dy
\end{equation}

\footnotesize
\begin{equation}
\Rightarrow \frac{-Ai}{2} \int_{-\infty}^{\infty} \int_{-\infty}^{\infty} (1) e^{i(2\pi u_0 x + 2\pi v_0 y)} e^{-i2\pi (ux + vy)} dx dy - \frac{-Ai}{2} \int_{-\infty}^{\infty} \int_{-\infty}^{\infty} (1) e^{-i(2\pi u_0 x + 2\pi v_0 y)} e^{-i2\pi (ux + vy)} dx dy
\end{equation}
\normalsize

and after applying the translation property of the Fourier Transformation, we then get:

\begin{equation}
\Rightarrow \frac{-Ai}{2} {\cal F} (1) (u - u_0, v - v_0) - \left( \frac{-Ai}{2} \right) {\cal F} (1) (u + u_0, v + v_0)
\end{equation}

\begin{equation}
\Rightarrow A\frac{i}{2}\Big[\delta(u+u_0,v+v_0)-\delta(u-u_0,v-v_0)\Big] \hspace{0.1in}.
\end{equation}

\item[4)] Assume that ${\cal F}(1)=\delta$ also holds in the discrete case (this can be shown). Using this property and the translation property, show that 
the Fourier transform of the discrete function $f(x,y)=\sin(2\pi u_0 x+2\pi v_0y)$ is 
$$F(u,v)=\frac{i}{2}\Big[\delta(u+Mu_0,v+Nv_0)-\delta(u-Mu_0,v-Nv_0)\Big] \hspace{0.1in}.$$

\textbf{Answer: }We first apply the discrete Fourier transform to the given function:

$${\cal F}[f(x,y)] = \sum_{x = 0}^{M-1} \sum_{y = 0}^{N - 1} \sin(2\pi u_0 x+2\pi v_0y) e^{-i2\pi \left( \frac{ux}{M} + \frac{vy}{N} \right)}$$

We then can express our sine function as exponentials with the following:
$$\sin\theta = \frac{e^{i\theta} - e^{-i\theta}}{2i}$$

which, we then apply to our equation:

\begin{equation}
\Rightarrow \frac{-i}{2} \sum_{x = 0}^{M-1} \sum_{y = 0}^{N - 1} \left[ e^{i(2\pi u_0 x+2\pi v_0y)} - e^{-i(2\pi u_0 x+2\pi v_0y)} \right] e^{-i2\pi \left( \frac{ux}{M} + \frac{vy}{N} \right)} \hspace{0.1in} .
\end{equation}

Thus, we then can write:

\begin{equation}
\Rightarrow \frac{-i}{2} \sum_{x = 0}^{M-1} \sum_{y = 0}^{N - 1} (1) e^{i(2\pi u_0 x+2\pi v_0y)} e^{-i2\pi \left( \frac{ux}{M} + \frac{vy}{N} \right)} - \left( \frac{-i}{2} \right) \sum_{x = 0}^{M-1} \sum_{y = 0}^{N - 1} (1) e^{-i(2\pi u_0 x+2\pi v_0y)} e^{-i2\pi \left( \frac{ux}{M} + \frac{vy}{N} \right)}
\end{equation}

Using the translation property, we can then write:

\begin{equation}
\Rightarrow \frac{-i}{2} {\cal F}(1) (u - Mu_0, v - Nv_0) - \left( \frac{-i}{2} \right) {\cal F}(1) (u + Mu_0, v + Nv_0)
\end{equation}

\begin{equation}
\Rightarrow \frac{i}{2}\Big[\delta(u+Mu_0,v+Nv_0)-\delta(u-Mu_0,v-Nv_0)\Big] \hspace{0.1in}.
\end{equation}

\item[5)] \textbf{Periodic Noise Reduction Using a Notch Filter}
\begin{enumerate}
\item[a)] Write a program that implements sinusoidal noise of the form given in the previous homework: $n(x,y)=A\sin(2\pi u_0 x+2\pi v_0y)$.  
The input to the program must be the amplitude, A, and the two frequency components $u_0$ and $v_0$.

\textbf{Answer:} The following is the code written to generate the noise of the image:

\begin{lstlisting}
original_image = imread('image.jpg');
original_image_double = im2double(original_image);

[M, N] = size(original_image_double);

A = input('Amplitude:');
frequency1 = input('Frequency 1:');
frequency2 = input('Frequency 2:');

for i = 1:M
    for j = 1:N
        noise(i,j) = A * sin(2 * pi * frequency1 * i + 2 * pi * frequency2 * j);
    end
end
\end{lstlisting}

\item[b)] Download image 5.26(a) of size $M\times N$ and add sinusoidal noise to it, with $v_0 = 0$.  The value of A must be high enough for the 
noise to be quite visible in the image (for example, you can take $A=100$, $u_0=134.4$, $v_0=0$). 

\textbf{Answer:} The following is the code written to add noise to the image:

\begin{lstlisting}
for k = 1:M
    for l = 1:N
        B(k,l) = original_image_double(k,l) + noise(k,l);
    end
end

figure, imshow(B), title('Degraded Image');
\end{lstlisting}

\newpage
\item[c)]  Compute and display the degraded image and its spectrum (you may need to apply a log transform to visualize the spectrum).  

\textbf{Please see attached pages for image and spectrum.}

\begin{lstlisting}
fourier_transform = fft2(shift);
spectrum = abs(fourier_transform);

for i = 1:M
    for j = 1:N
        plot_spectrum(i,j) = 5 * log(1 + spectrum(i,j));
    end
end
\end{lstlisting}

\item[d)] Notch-filter the image using a notch filter of the form shown in Fig. 5.19(c), to remove the periodic noise. 

\textbf{Please see attached pages for image and spectrum.} \newline Note: Some of the functions given in the code were pulled from StackOverflow.

\begin{lstlisting}
fourier_transform2 = fftshift(fft2(original_image));
norm_img = @(img) (img - min(img(:))) / (max(img(:)) - min(img(:)));
gNotch = @(v,mu,cov) 1-exp(-0.5*sum((bsxfun(@minus,v,mu).*(cov\bsxfun(@minus,v,mu)))));
center_x = 129;
center_y = 129;

% distance of noise from center
wx1 = 149.5-129;
wx2 = 165.5-129;
wy  = 157.5-129;

% create notch filter
notch_filt = ones(M,N);

[y,x] = meshgrid(1:N, 1:M);
X = [y(:) x(:)].';
notch_filt = notch_filt .* reshape(gNotch(X,[center_x+wx1;center_y+wy],eye(2)*25),[M,N]);
notch_filt = notch_filt .* reshape(gNotch(X,[center_x+wx2;center_y+wy],eye(2)*25),[M,N]);
notch_filt = notch_filt .* reshape(gNotch(X,[center_x-wx1;center_y-wy],eye(2)*25),[M,N]);
notch_filt = notch_filt .* reshape(gNotch(X,[center_x-wx2;center_y-wy],eye(2)*25),[M,N]);

% apply filter
fourier_transform2 = fourier_transform2 .* notch_filt;

% compute inverse
ifft_ = ifft2(ifftshift(fourier_transform2));
restored_image = histeq(norm_img(ifft_));
\end{lstlisting}

\end{enumerate}

\end{enumerate}
\end{document}